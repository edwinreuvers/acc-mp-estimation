\begin{tabular}{|l|c|c|c|c|c|c|c|c|} \hline
 & \multicolumn{4}{c|}{\itshape Traditional method} & \multicolumn{4}{c|}{\itshape Improved method} \\ \hline
 & \bfseries QR & \bfseries SR & \bfseries ISOM & \bfseries SSC & \bfseries QR & \bfseries SR & \bfseries ISOM & \bfseries SSC \\ \hline
Rat 1 & 5.6 ± 3.4 & 5.6 ± 0.8 & 6.8 ± 3.2 & 6.2 ± 3.0 & 3.3 ± 1.5 & 2.8 ± 0.9 & 5.4 ± 3.0 & 5.4 ± 3.1 \\ \hline
Rat 2 & 4.2 ± 1.5 & 5.0 ± 0.6 & 4.3 ± 1.6 & 4.6 ± 2.2 & 2.7 ± 1.6 & 2.1 ± 0.5 & 4.2 ± 2.6 & 4.2 ± 2.0 \\ \hline
Rat 3 & 6.4 ± 2.3 & 6.8 ± 1.0 & 6.1 ± 2.1 & 5.5 ± 2.7 & 4.2 ± 1.8 & 2.6 ± 0.5 & 4.9 ± 2.3 & 5.0 ± 2.7 \\ \hline
Avg ± Std & 5.4 ± 2.6 & 5.8 ± 1.1 & 5.7 ± 2.7 & 5.4 ± 2.7 & 3.4 ± 1.8 & 2.5 ± 0.7 & 4.8 ± 2.7 & 4.9 ± 2.7 \\ \hline
\end{tabular}\break\hfill\footnotesize{Root mean squared differences are expressed as a percentage of the maximal isometric CE force ($F_{CE}^{max}$). Root mean squared differences were computed over the interval in which CE stimulation was maximal for the quick-release and step-ramp experiments and was computed over the interval from CE stimulation onset to 0.1 s after CE stimulation offset for the isometric experiments and the stretch-shortening cycles.}